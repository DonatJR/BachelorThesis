\chapter{Implementierung eines Prototypen}\label{chap:implementation}
In diesem Kapitel wird die Entwicklung eines auf dem zuvor ausgearbeiteten Konzept basierten Prototypen beschrieben. Dieser Prototyp enthält nicht alle im Konzept beschriebenen Anforderungen und befindet sich auch nicht in einem finalen Entwicklungsstadium, kann aber mit weiteren Entwicklungsressourcen als Grundlage für eine finale Implementierung genutzt werden. Er soll zeigen, wie sich das entworfene Konzept umsetzen lässt und die darin benutzten Technologien miteinander interagieren.

\section{Tool zum Parsen der vorhandenen UI-Struktur}
Wie in Kapitel~\ref{chap:concept} beschrieben, müssen die Dateien, welche das momentane UI-Layout enthalten, in ein web-freundliches Format (JSON) übersetzt werden. Während dieses Prozesses können künftig nicht mehr benötigte Informationen übergangen, das heißt nicht mit in das neue Format übernommen, und relevante Informationen direkt in eine optimierte Struktur überführt werden. Für diesen Zweck wurde ein kleines Hilfstool in \nameformat{C\#} geschrieben, welches sowohl die \enquote{.dli}-Datei der Detailansicht als auch die \enquote{.vlc}-Datei der Übersichtsliste einer einzelnen \nameformat{\gls{crm}}-Ansicht als Input erhält und daraus eine \enquote{.json}-Datei mit allen benötigten Informationen erstellt. Um die Anpassbar- und Wiederverwendbarkeit des Tools zu maximieren, wurde das \textbf{Visitor-Pattern} angewandt. In Abbildung~\ref{fig:web-conv_file-tree} ist die Datei-Struktur des Tools dargestellt.

\begin{figure}
    \centering
    \captionsetup{justification=centering}
    \frame{\includegraphics[width=0.3\textwidth]{figures/web-converter_file-tree.png}}
        \caption{Datei-Struktur des Konvertierungstools}\label{fig:web-conv_file-tree}
\end{figure}

Zunächst wird für jedes einzulesende Token (XML-Element) eine Klasse vom \nameformat{Acceptor}-Interface (\nameformat{IDLIAcceptor} / \nameformat{IVLCAcceptor}) abgeleitet, dieses Interface enthält eine einzige Methode \nameformat{Apply}, welche einen Visitor (\nameformat{IDLIVisitor} / \nameformat{IVCLVisitor}) übergeben bekommt. Die jeweiligen Token-Klassen werden mit dem von ihnen verwalteten XML-Element (\nameformat{XElement}) initialisiert. Wie im Quellcodeauszug~\ref{lst:pageelement_init} anhand der Klasse \nameformat{PageElement} beispielhaft zu sehen, werden aus diesem XML-Element die relevanten Informationen über das Element selbst (Name und Metadaten) und dessen verschachtelte Kinder-Elemente (Liste von Controls) ausgelesen.

\lstinputlisting[language={[Sharp]C},label={lst:pageelement_init},caption={Initialisierung der PageElement-Klasse}]{code/chapter_005_pageelement_init.cs}

Nachdem die Informationen der XML-Datei auf diese Art und Weise in ihre einzelnen Token-Instanzen übersetzt wurden, wird die \nameformat{Apply}-Methode, zu sehen in Quellcodeauszug~\ref{lst:dialogelement_apply}, des zentralen Tokens (\nameformat{DialogElement}) mit einer Visitor-Instanz aufgerufen. Der Visitor erhält als Parameter eben diese Instanz und extrahiert alle für ihn relevanten Informationen. Anschließend ruft er rekursiv die \nameformat{Apply}-Methoden der Kinder-Token auf und liest auch aus diesen die relevanten Informationen aus. Diese Aufrufe sind im Quellcodeauszug in Anhang~\ref{lst:appendix_json-visitor_visit-methods} zu sehen. Nachdem alle Tokens vollständig besucht wurden, können die gewonnenen Daten als JSON-String ausgegeben werden.

\lstinputlisting[language={[Sharp]C},label={lst:dialogelement_apply},caption={\nameformat{Apply}-Methode der DialogElement-Klasse}]{code/chapter_005_dialogelement_apply.cs}

Die Flexibilität dieser Architektur, welche in der Abbildung~\ref{fig:web-conv_class-diagramm} nochmals übersichtlich als Klassendiagramm dargestellt wird, ist des Weiteren daran zu erkennen, dass der einzige Unterschied beim Auslesen von Detailansicht-Datei und Übersichtslisten-Datei in der Implementierung der Interfaces besteht. Sowohl \nameformat{Acceptor}- als auch \nameformat{Visitor}-Klassen können sehr leicht einzeln angepasst oder ersetzt werden. Ebenso besteht die Möglichkeit, weitere Tokens, welche eventuell zu einem späteren Zeitpunkt benötigt werden, aus den XML-Dateien auszulesen, indem weitere \nameformat{Acceptor}- und \nameformat{Visitor}-Implementierungen hinzufügt werden.

\begin{figure}[htb]
    \centering
    \captionsetup{justification=centering}
    \includegraphics[width=0.9\textwidth]{figures/web-converter_class-diagramm.png}
        \caption{Klassendiagramm des Visitor-Patterns im Konvertierungstool}\label{fig:web-conv_class-diagramm}
\end{figure}

Der Input und das Endergebnis in Form eines vom \nameformat{GraphQL}-Server direkt verwertbaren JSON-Dokuments sind in Auszug~\ref{lst:xml_input} und~\ref{lst:json_output} anhand einer gekürzten Darstellung ersichtlich.

\lstinputlisting[language={XML},caption={XML-Input für das Konvertierungstool},label={lst:xml_input}]{code/chapter_005_xml_input.xml}

\lstinputlisting[caption={JSON-Ergebnis des Konvertierungstools},label={lst:json_output}]{code/chapter_005_json_output.json}

\section{Umsetzung der Webseite}
Das Projekt für die Webseite wurde mit dem von \nameformat{Facebook} entwickelten \gls{cli}-Tool \nameformat{\gls{cra}} erstellt. Dieses führt die allgemeine Konfiguration, die Einrichtung von \nameformat{TypeScript} und der Test-Funktionalitäten selbstständig durch, weitere Anpassungen sind daher nicht notwendig. Neben den in diesem Abschnitt beschriebenen Komponenten und Tests enthält das Projekt ausschließlich die Definition zweier verschiedener \nameformat{CSS}-Themes und einige wenige temporäre UI-Elemente zur Steuerung des Verhaltens (Umschalten des Anzeigemodus, des Themes und des Editier-Modus).

\subsection{Erstellung der React-Komponenten}
Der Aufbau aller Komponenten folgt einem identischen Schema. Zunächst ist eine \nameformat{React}-Komponente so definiert, dass sie als Parameter alle notwendigen Eigenschaften zur Anzeige und Mutation von Daten übergeben bekommt. Die Komponente enthält keine Logik zur Veränderung der Parameter.

\nameformat{CSS} wird mithilfe der Bibliothek \nameformat{styled-components} in Form eines Wrappers um das eigentlich anzuzeigende \nameformat{HTML}-Tag eingesetzt. Ein Nachteil des traditionellen \nameformat{CSS}-Einsatzes ist, dass \nameformat{CSS}-Klassen an vielen Stellen der Oberfläche eingesetzt werden und eine Änderung am Aussehen durch den fehlenden Überblick über sämtliche Stellen möglicherweise ungewollte Auswirkungen haben kann. Durch die Verknüpfung von \nameformat{CSS} mit \nameformat{React}-Komponenten kann sich eine Änderung immer nur auf die jeweils verknüpfte Komponente auswirken. In Auszug~\ref{lst:styled_components} ist die Benutzung und Unterstützung von Themes von \nameformat{styled-components} anhand eines \nameformat{HTML} input-Tags dargestellt.

\lstinputlisting[label={lst:styled_components},caption={Benutzung der \nameformat{styled-components} Bibliothek}]{code/chapter_005_styled_components.js}

Eine Komponente ist nicht selbst für das Abschicken der \nameformat{GraphQL}-Abfragen, welche die Struktur der Ansicht liefern, verantwortlich. Dennoch wird der jeweilige Teil der Abfrage mit den für sie relevanten Daten in der Projektstruktur zusammen mit der\\ Komponente definiert. Eine speziell dafür ausgelegte Komponente in einem höheren Teil der Hierarchie kombiniert sämtliche Teilabfragen der UI-Komponenten (siehe Quellcodeauszug~\ref{lst:graphql_query}) und sendet diese an den Server. Diese Komponente ist weiterhin dafür zuständig, die vom Server erhaltenen Daten wieder auf die UI-Komponenten zu verteilen. Für die Abfragen der tatsächlichen Daten wird eine weitere Abfrage für jede Komponente definiert, welche jeweils auch von dieser Komponente ausgeführt und ausgewertet wird.
Der vierte und letzte Bestandteil einer Komponente sind die im nächsten Abschnitt beschriebenen Tests.

\lstinputlisting[label={lst:graphql_query},caption={Einbettung von \nameformat{GraphQL}-Query-Fragmenten}]{code/chapter_005_graphql_query.gql}

\subsection{Integration der Tests}
Einzelne Komponenten werden durch die in Abschnitt~\ref{subsec:test_ci_concept} erwähnten Snapshot-Vergleiche (siehe Quellcodeauszug~\ref{lst:snapshot_test}) auf unerwünschte Veränderungen geprüft. Weiterhin wird getestet, ob die Übergabeparameter in Form von Formatierungsbedingungen oder anzuzeigenden Daten korrekt ausgewertet und dargestellt werden. Da die Clientapplikation bewusst wenig Logik enthält, werden in diesem Bereich auch verhältnismäßig wenige Tests benötigt.
Die während der Entwicklung genutzten visuellen \nameformat{Storybook}-Tests müssen für jede einzelne Komponente konfiguriert werden. Notwendig ist lediglich die Konfiguration der Komponente selbst. Optional werden aber jeweils noch die im Produktivcode ebenso benutzten globalen \nameformat{CSS}-Deklarationen, die verfügbaren Themes und die in der UI veränderbaren Übergabeparameter (im Auszug \enquote{Knobs}) angegeben. Abbildung~\ref{fig:storybook_example} zeigt die Darstellung der in Quellcodeauszug~\ref{lst:storybook_config} gezeigten Konfiguration in der \nameformat{Storybook}-Umgebung.

\lstinputlisting[language={JavaScript},caption={Snapshot-Test mit \nameformat{Jest} und \nameformat{Enzyme}},label={lst:snapshot_test}]{code/chapter_005_snapshot_test.js}

\lstinputlisting[language={JavaScript},caption={Konfiguration einer \nameformat{Storybook}-Komponente},label={lst:storybook_config}]{code/chapter_005_storybook_config.js}

\begin{figure}
    \centering
    \captionsetup{justification=centering}
    \includegraphics[width=\textwidth]{figures/storybook_example.png}
        \caption{Beispiel für die Komponentendarstellung mit \nameformat{Storybook}}\label{fig:storybook_example}
\end{figure}

\subsection{GraphQL-Mock-Server und Resolver}
Da das passende Backend zu dieser Arbeit erst zu einem späteren Zeitpunkt entwickelt wird, wurde vorübergehend eine Dummy-Implementierung eines \nameformat{GraphQL}-Servers erstellt, welcher aber bereits das korrekte Schema und die Daten des erstellten Parser-Tools nutzt. Dies ermöglicht es, den Client unter realen Gegebenheiten zu entwickeln. Die Implementierung besteht aus der Schemadefinition für \nameformat{GraphQL} (ein Teil davon ist in Quellcodeauszug~\ref{lst:graphql_schema} zu sehen) und den Resolvern, welche dem Schema die konkreten Daten zuordnen (der zu Auszug~\ref{lst:graphql_schema} passende Code kann in Auszug~\ref{lst:graphql_schema_resolver} eingesehen werden). Aus diesen beiden Einzelteilen wird mithilfe der \nameformat{Apollo GraphQL-tools}~\parencite*{apollo_graphql-tools_2019} ein ausführbares Schema erstellt und mit \nameformat{express-graphql}~\parencite{express_graphql_2018} lokal gehostet. Für die im Schema definierten Typen, welche aufgrund fehlender Daten noch keine Resolver erhalten können, werden von den \nameformat{GraphQL-tools} automatisch Daten mit passenden Typen generiert.

\lstinputlisting[label={lst:graphql_schema},caption={Teil der \nameformat{GraphQL}-Schemadefinition}]{code/chapter_005_graphql_schema.gql}

\lstinputlisting[language={JavaScript},label={lst:graphql_schema_resolver},caption={\nameformat{GraphQL} Schema-Resolver}]{code/chapter_005_graphql_schema_resolver.js}

\section{Beispiel anhand einer simplen Oberfläche}
Um die Validität des Konzepts zu belegen, wird eine beispielhafte, simple Oberfläche des \nameformat{\gls{crm}} Desktopclients in die neue Darstellung übersetzt. In den Abbildungen~\ref{fig:listview_crm} und~\ref{fig:listview_new} ist die bisherige sowie die neue Übersichtsliste zu sehen. Die angezeigten Spalten der neuen Liste entsprechen dabei der Konfiguration der Liste auf dem Desktopclient. 

\begin{figure}[htb]
    \centering
    \captionsetup{justification=centering}
    \includegraphics[width=0.9\textwidth]{figures/listview_crm.png}
        \caption{Übersichtsliste des Desktopclients}\label{fig:listview_crm}
\end{figure}

\begin{figure}[htb]
    \centering
    \captionsetup{justification=centering}
    \includegraphics[width=0.9\textwidth]{figures/listview_new.png}
        \caption{Übersichtsliste der Weboberfläche}\label{fig:listview_new}
\end{figure}

Analog zur Übersichtsliste stellen die Abbildungen~\ref{fig:detailview_crm} und~\ref{fig:detailview_new} die Detailansicht des Desktopclients und der Web-UI (mit aktiviertem \enquote{Dark Mode}) dar. Das Layout der Weboberfläche entspricht unmittelbar nach der automatisierten Umsetzung einer Liste von nacheinander angezeigten Elementen. Sie wurde für die Darstellung daher bereits\\ manuell angepasst, um der des Desktops zu entsprechen.

\begin{figure}[htb]
    \centering
    \captionsetup{justification=centering}
    \includegraphics[width=0.9\textwidth]{figures/detailview_crm.png}
        \caption{Detailansicht des Desktopclients}\label{fig:detailview_crm}
\end{figure}

\begin{figure}[htb]
    \centering
    \captionsetup{justification=centering}
    \includegraphics[width=0.9\textwidth]{figures/detailview_new.png}
        \caption{Detailansicht der Weboberfläche (\enquote{Dark Mode})}\label{fig:detailview_new}
\end{figure}

Sämtliche dargestellten Daten beider Ansichten, inklusive der generierten Dummy-Daten, wurden vom Mock-Server abgefragt, somit hat die Weboberfläche keinerlei Kenntnisse über die Daten. Die neuen Oberflächen enthalten noch einige zusätzliche Darstellungselemente, etwa die Checkboxen zum Wechseln des Modus von Übersichtsliste zur Detailansicht und Aktivieren des \enquote{Dark Mode}. Auch die Informationen zu den Elementtypen in der Detailansicht werden in der finalen Umsetzung nicht enthalten sein, durch die Anzeige von Mock-Daten ist es andernfalls jedoch schwierig zu erkennen, welche Daten zu welchem Feld gehören.
