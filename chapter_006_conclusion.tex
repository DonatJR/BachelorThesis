\chapter{Fazit}\label{chap:conclusion}
Das ursprüngliche Ziel, die Umsetzung der gesamten UI, war in der zur Verfügung stehenden Zeit nicht realisierbar. Der Grund dafür ist, dass der Desktopclient des \nameformat{\gls{crm}} zu viele UI-Elemente besitzt und die zu beachtenden Aspekte für deren Umsetzung zu zahlreich sind, um diese in einer angemessenen Qualität bearbeiten zu können. Eine Modifikation der Ziele hin zur Erstellung eines Konzepts für die dynamischen Teile der \nameformat{\gls{crm}} UI und das Erstellen eines simplen Prototypen war damit angebracht. Auf die Bearbeitung der statischen Elemente der UI, die Benutzerverwaltung, die Berichts- und Webansicht, die Termin- und Aufgabenplanung und Weitere wurde somit verzichtet. Die modifizierten Ziele konnten erreicht werden.

Die Entwicklung mit \nameformat{React} ist unkompliziert, die Auswahl an hilfreichen Bibliotheken groß und in der Community können zu jeder Fragestellung passende Antworten gefunden werden. \nameformat{React} kann daher im folgenden Prozess weiterhin ohne Bedenken eingesetzt werden. Für ein qualifiziertes Fazit über die Nutzung von \nameformat{GraphQL} reicht die bisher gesammelte Erfahrung nicht vollständig aus --- es lässt sich jedoch festhalten, dass die Technologie sich im Rahmen dieses Projekts positiv bewährt hat. Eine weitere Evaluation von \nameformat{GraphQL}, insbesondere auch in größeren Projekten, scheint somit empfehlenswert.

Insgesamt hat die Umsetzung des Konzepts gezeigt, dass es funktioniert. Der Ausbau ebendieses Konzepts mit den bisher nicht beachteten Elementen ist jedoch unabdingbar. Der Prototyp dient hierfür als solide Basis für die weitere Entwicklung. Zusätzliche Zeit wäre daher besonders wünschenswert gewesen, nicht zuletzt, um mehr Erfahrungen mit Anforderungsformulierungen und den Technologien zu sammeln sowie eine fundiertere Wissensbasis, insbesondere in den Bereichen des API-Designs und über \nameformat{GraphQL}-Server, für die nächsten Schritte zu kumulieren.

\section{Aufgetretene Schwierigkeiten}
Wie oben erwähnt, wurde das ursprüngliche Ziel, die Umsetzung der gesamten UI, nicht erreicht. Obwohl es sich dabei nicht um eine technische Schwierigkeit handelte, hatte sie dennoch weitreichende Auswirkungen auf die letztendliche Umsetzung. Aufgrund dessen wurden zahlreiche, zu Beginn der Arbeit, antizipierten Schwierigkeiten, insbesondere die serverbasierte Persistenz von Einstellungen und die Entscheidung über die im Web umsetzbaren Features, nicht Gegenstand der Problemanalyse sowie der Problemlösung.

Des Weiteren wurde während der Evaluation der Frontend-Frameworks klar, dass es in diesem Bereich derart viele Ressourcen zu evaluieren gibt, dass eine vollständige Analyse dieser mehr als das Arbeitspensum eines einzelnen in Anspruch nehmen würde. Auf der anderen Seite sind sich die Technologien in ihrer Funktionalität aber dennoch so ähnlich, dass eine Entscheidung zum Großteil basierend auf persönlichen und betrieblichen Präferenzen getroffen werden musste. Trotz dessen eine möglichst objektive und zufriedenstellende Entscheidung für eine Technologie zu treffen, kostete mehr Zeit und Anstrengung als erwartet.

Eine weitere Schwierigkeit bestand darin, das gesetzte Ziel nicht aus den Augen zu verlieren. Ein Konzept für ein technisches Projekt umfasst zahlreiche Themengebiete und noch mehr Entscheidungen, welche sich zu einem unbestimmten Zeitpunkt in der Zukunft als Fehler erweisen könnten. Sämtliche Anforderungen müssen jederzeit bedacht werden und präsent sein. Zudem ist es notwendig Lösungen zu finden, welche all diesen Anforderungen gleichzeitig gerecht werden oder zumindest keine negativen Auswirkungen auf andere Bereiche nach sich ziehen. Diese mentale Belastung ist mitunter sehr anstrengend und kann bei der Entscheidungsfindung lähmen, wodurch die Erreichung des Ziels erschwert wurde.

\section{Ausblick}
In diesem Abschnitt wird ein Ausblick auf zukünftige Entwicklungsmöglichkeiten zum weiteren Ausbau des Projektes gegeben. Zunächst sollen die noch nicht fertiggestellten Funktionalitäten abgeschlossen werden. Dazu zählen zwei \nameformat{React}-Komponenten, ein Tab- und Container-Element, die Vervollständigung der Formatierungsmöglichkeiten und der Implementierung von Lade- und Fehlerzuständen aller Komponenten.
Im Anschluss oder parallel kann weiterhin an einer verbesserten Umsetzung der \nameformat{GraphQL}-API und auch der Abfragen geforscht werden. Die geringe Erfahrung zu Beginn der Arbeit in diesem Bereich lässt darauf schließen, dass nicht alle Implementierungen optimal gelöst wurden und weiteres Verbesserungspotential besteht.

Sobald diese Aufgaben zufriedenstellend abgeschlossen sind, können die weiteren Aspekte der Anforderungsanalyse in Angriff genommen werden. Zum einen sind dies sämtliche weiteren Bestandteile der UI (inklusive der funktionalen Anforderungen an Web-Apps) und zum anderen das dafür benötigte Backend mit einem \nameformat{GraphQL}-Server, der mit Echtdaten arbeitet.
