\chapter{Technologien}

\section{Entwicklungsumgebung} % Sprache, ide, tests, CI, ... auch für Tool zur automatischen Übersetzung!
\subsection{Typescript}
- OS Sprache von Microsoft
- syntaktische Obermenge von zu JS
- Optionale statische Typen für JavaScript
   -> Typsicherheit
   -> Codevervollständigung 
- wird zu JavaScript transpiliert
   -> es können automatisch die neusten Features von JS (ES6+) benutzt werden 
- mehr Sicherheit und bessere Unterstützung durch Entwicklertools beim Programmieren
- viele Fehler die in JS erst zur Laufzeit auffallen und schwer nachvollziehenden Problemen führen werden bereits beim Kompilieren gefunden
- Nachteil nur der zusätzliche compile Schritt bzgl. Tooling / Aufwand

% Auswahl Frameworks:  Why popularity does matter
% https://www.npmjs.com/npm/state-of-javascript-frameworks-2017-part-1

% weiteres Kriterium: Typescript Kompatibilität


\section{Frontend-Frameworks}
% Komponenten in dli anschauen und Übersetzungen für alle überlegen, dann beispielhaft von hand für verschiedene frameworks umsetzen und vergleichen.

% => nicht nur Elemente sondern auch Anordnung wichtig!!

  %      -> Vergleich & für eins entscheiden: React, Vue.js, Angular, Polymer, ? ... ?, Ember, Knockout, Riot
   %     -> warum nicht: bootstrap, semantic-UI, foundation, materialize, material UI, pure, skeleton, UI kit, milligram, Susy

% job market important in deciding for framework

\section{REST (OData) vs. GraphQL}
