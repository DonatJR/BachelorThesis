% document setup
\documentclass[12pt, twoside, a4paper, openright]{report}

% omits page numbering on empty pages
\usepackage{emptypage}

% page numbering and chapter name in header
% to get rid of footer page numbers on new chapter pages use
% \thispagestyle{empty} after the \chapter command
% for more information: https://ftp.uni-erlangen.de/ctan/macros/latex/contrib/fancyhdr/fancyhdr.pdf
\usepackage{fancyhdr}
\fancyhf{} % empties header and footer
\pagestyle{fancy} % use the fancy style for all pages
\renewcommand{\chaptermark}[1]{\markboth{#1}{}} % chapter name (\leftmark) should not show chapter number and should also not be uppercase
\renewcommand{\sectionmark}[1]{\markright{#1}}
% \renewcommand{\headrulewidth}{0pt} % and the line % defines the height of the line of the header
\fancyhead[LE,RO]{\thepage} % LE = left even, RO = right odd, \thepage shows page numbering
\fancyhead[LO]{\rightmark} % \rightmark shows name of current subsection (\leftmark for section)
\fancyhead[RE]{\leftmark} % \leftmark shows name of current chapter / section (\rightmark for subsection)
\setlength{\headheight}{15pt}

\usepackage[left=3cm, 
			right=2.5cm, 
			top=2.5cm, 
			bottom=2.5cm, 
			includehead, 
			includefoot]{geometry}

% language /font setup
\usepackage{fontspec}
\usepackage{polyglossia}
\usepackage{csquotes}
\setmainlanguage[babelshorthands=true]{german}
\setmainfont{Cambria}
            
% linespacing
\usepackage{setspace}
\setstretch{1.25}

% special characters
\usepackage{amssymb}

% simple math packages
\usepackage{amsfonts}
\usepackage{amsmath}
\usepackage{mathtools} % e.g. text over arrows
\usepackage[d]{esvect} % for vector arrows

% coloring
\usepackage[dvipsnames]{xcolor}
\definecolor{LightCyan}{rgb}{0.88,1,1}

% european comma delimiter (space after comma dependent on context)
\usepackage{icomma} 

% -------------------------------------------
% text boxes
% -------------------------------------------
\usepackage{mdframed}
\usepackage{ntheorem}

% text box for definitions
\newmdtheoremenv[%
  backgroundcolor=gray!40,
  linewidth=1pt,
  topline=false,
  rightline=false,
  leftline=false,
  bottomline=false]{Definition}{Definition}
  
% text box for Beispiel
\newmdtheoremenv[%
  backgroundcolor=gray!0,
  linewidth=1pt,
  topline=false,
  rightline=false,
  bottomline=false]{Beispiel}{Beispiel}

% text box for Element
\theoremstyle{nonumberplain}
\newmdtheoremenv[%
  backgroundcolor=gray!30,
  linewidth=1pt,
  topline=false,
  rightline=false,
  leftline=false,
  bottomline=false]{Element}{Einordnung Element}
\theoremstyle{plain}
  
% text box for Menge
\theoremstyle{nonumberplain}
\newmdtheoremenv[%
  backgroundcolor=gray!30,
  linewidth=1pt,
  topline=false,
  rightline=false,
  leftline=false,
  bottomline=false]{Menge}{Einordnung Menge}
\theoremstyle{plain}
  
% text box for Funktion
\theoremstyle{nonumberplain}
\newmdtheoremenv[%
  backgroundcolor=gray!30,
  linewidth=1pt,
  topline=false,
  rightline=false,
  leftline=false,
  bottomline=false]{Funktion}{Einordnung Funktion}
\theoremstyle{plain}
 
% text box for Element
\theoremstyle{nonumberplain}
\newmdtheoremenv[%
  backgroundcolor=gray!30,
  linewidth=1pt,
  topline=false,
  rightline=false,
  leftline=false,
  bottomline=false]{Standardform}{Standardform}
\theoremstyle{plain}
  
\theoremstyle{nonumberplain}
\newmdtheoremenv[%
  backgroundcolor=gray!40,
  linewidth=1pt,
  topline=false,
  rightline=false,
  leftline=false,
  bottomline=false]{DualTheorem}{Dualitätstheorem der linearen Optimierung}
\theoremstyle{plain}

% text box for Lemma
\newmdtheoremenv[%
  backgroundcolor=gray!40,
  linewidth=1pt,
  topline=false,
  rightline=false,
  leftline=false,
  bottomline=false]{Lemma}{Lemma}  
  
% text box for Theorem
\newmdtheoremenv[%
  backgroundcolor=gray!40,
  linewidth=1pt,
  topline=false,
  rightline=false,
  leftline=false,
  bottomline=false]{Theorem}{Theorem}
% -------------------------------------------

% citations
\usepackage[style=alphabetic,isbn=false]{biblatex}
\bibliography{../content/literatur}

% tables and coloring of tables
\usepackage{colortbl}
\usepackage{tabularx}
\addto\captionsgerman{\renewcommand{\tablename}{Tab.}}
\newcolumntype{a}{>{\columncolor{lightgray!30}}c}
\newcolumntype{b}{>{\columncolor{lightgray!30}}l}

% Euro-symbol
\usepackage[official]{eurosym}

% paragraph setup
\usepackage{changepage} % to indent a whole paragraph
\setlength\parindent{0pt} % no indent after each paragraph

% graphics
\usepackage{graphicx,caption}
\usepackage{enumitem} % for graphics as items
\addto\captionsgerman{\renewcommand{\figurename}{Abb.}}

% glossary
\usepackage[toc,nonumberlist,nopostdot=true]{glossaries}
\makeglossaries %
% label, short form, long form: use with \gls{ma}
\newacronym{D}{D}{Duales Optimierungsproblem}
\newacronym{MA}{MA}{Mindestanforderungen}
\newacronym{NB}{NB}{Nebenbedingung}
\newacronym{P}{P}{Primales Optimierungsproblem}
\newacronym{S}{S}{Standardform}

% appendices (with pretty matlab code)
\usepackage[toc,page]{appendix}
\usepackage[framed,numbered]{matlab-prettifier}

% include complete pdf documents in tex document
\usepackage{pdfpages}

% margin between footnotes and text
\setlength{\skip\footins}{0.8cm}

% border for images
\usepackage[export]{adjustbox}

% better footnote formatting
\usepackage[para,bottom]{footmisc}

% don't insert space between elements to keep pages the same height
\raggedbottom