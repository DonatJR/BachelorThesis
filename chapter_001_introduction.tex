\chapter{Einleitung}\label{chap:introduction}

\section{Kapitelübersicht}
Dieses Kapitel führt in die Thematik ein, erläutert die Motivation der Arbeit und gibt einen Überblick über im Voraus antizipierte Schwierigkeiten bezüglich der Umsetzung des Architekturkonzepts.
In Kapitel~\ref{chap:requirements} werden Anforderungen (auch anhand bisheriger Features) ermittelt.
In Kapitel~\ref{chap:technologies} werden die in Betracht gezogenen Technologien und deren Zusammenspiel untereinander vorgestellt. Dazu gehören die zu nutzende Programmierumgebung (Sprache, Testframework, Continuous Integration), der Vergleich verschiedener Frontend-Frameworks die anhand der in Kapitel~\ref{chap:requirements} formulierten Anforderungen bewertet werden und die Protokolle zur Kommunikation mit dem Backend.
In Kapitel~\ref{chap:concept} wird die Ausarbeitung des Konzepts mit den in Kapitel~\ref{chap:technologies} ausgesuchten Technologien und der Aufbau der API zur Kommunikation mit dem Backend beschrieben.
In Kapitel~\ref{chap:implementation} wird der Entwicklungsprozess der beispielhaften Umsetzung des erstellten Konzepts dargestellt.
In Kapitel~\ref{chap:conclusion} wird das Ergebnis kritisch hinterfragt, die tatsächlich aufgetretenen Schwierigkeiten mit den zuvor Erwarteten verglichen und ein Ausblick auf weitere Entwicklungsschritte gegeben.

\section{Motivation}
Ein Großteil der heute genutzten Software kann über eine Weboberfläche angesprochen werden. Dies hat den Vorteil, dass keine langwierigen und komplizierten Installationen vorgenommen werden müssen und, eine bestehende Internetverbindung vorausgesetzt, die Applikation von überall erreichbar ist. Die Firma \nameformat{combit GmbH} hat diese Entwicklung früh erkannt und bietet für den \nameformat{\gls{crm}} eine entsprechende Weboberfläche an. Diese existiert bereits seit vielen Jahren und basiert, bedingt durch den damaligen Stand der Technik, auf nicht mehr zeitgemäßen Technologien. Da der Fokus der Entwicklung --- bestimmt durch die Bedürfnisse der Kunden --- auf der Desktopanwendung liegt, waren nicht mehr genügend personelle Ressourcen für den allmählichen Ersatz der Technologien durch moderne Alternativen vorhanden.
Webtechnologien bringen nicht nur Vorteile für ihre Nutzer mit sich, sondern beinhalten auch einige interessante Aspekte für das Entwicklerteam selbst. So kann dieses moderne Projekte und Konzepte, welche durch die vielen technischen Fortschritte in diesem Bereich entstanden sind, kennenlernen und hat eine breite Auswahl an bestehenden Projekten für eine solide Basis des eigenen Produkts. Ein Beispiel für einen solchen Fortschritt sind Programme, die es erlauben, Webseiten wie normale Programme nativ auf einem Desktopcomputer darstellen zu können. Das Ziel hierbei ist, langfristig nur noch eine einzige Version für alle Umgebungen entwickeln zu müssen, wodurch Ressourcen gespart werden können. Durch den einheitlichen Fokus, so die Erwartung, soll zudem ein besseres Produkt entsteht. Dieses Potential der Webtechnologien wird dadurch noch weiter gesteigert, dass durch einfache Sprachen wie HTML, CSS und JavaScript, welches nicht kompiliert werden muss und auf der Mehrheit der modernen Geräte lauffähig ist, schnell viele Iterationen der Software erstellt werden können.
Wegen der genannten Gründe wurde entschieden, das Thema Web mit einer innovativen Herangehensweise, einer veränderten Perspektive und modernen Technologien neu anzugehen um den \nameformat{\gls{crm}} sowohl extern als auch intern noch attraktiver zu gestalten.

\section{Erwartete Schwierigkeiten}
Bereits im Vorfeld soll ein kleiner Überblick über die Aspekte dieser Arbeit gegeben werden, welche im Laufe der Umsetzung zu Schwierigkeiten führen könnten. Die vorherige Auseinandersetzung mit eventuell auftretenden Problemen ermöglicht die Antizipation ebendieser und somit eine vorausschauendere Auseinandersetzung mit der Aufgabe. Zudem dient sie der Reflexion der eigenen Fähigkeiten und damit deren Ausbau. Weitere Informationen zu den hier aufgelisteten Punkten werden in Kapitel~\ref{chap:requirements} aufgeführt.

Eine der größten Herausforderungen wird sein, eine Umsetzung des User Interface zu finden, welche auch weiterhin ebenso oder zumindest annähernd anpassbar ist wie dies in der aktuellen Version der Fall ist. Da das Layout dateibasiert gespeichert wird und im Browser nicht auf diese Dateien zugegriffen werden kann, muss ein neues Konzept für die Persistenz erarbeitet werden, etwa die Speicherung in einer Datenbank, mithilfe dessen die Verteilung über das Netzwerk einfacher möglich ist. 
Neben dem angesprochenen Layout gibt es auch noch weitere Informationen, die momentan nur dateibasiert auf dem jeweiligen Client gespeichert sind. Diese müssen ebenfalls alle in ein neues Konzept überführt werden.

Um auf alle diese Informationen sowie die eigentlichen Daten, die dem Nutzer angezeigt werden sollen, zugreifen zu können, wird eine entsprechende Schnittstelle (API) benötigt. Das Design dieser API ist kritisch, da sich nachträglich nötige Änderungen auf jede Komponente auswirken, was wiederum Anpassungen an ebendiesen erfordert.

Weiterhin gilt es zu entscheiden, welche Features im Web überhaupt nicht oder nur mit zu vielen Schwierigkeiten, etwa eine auf Desktopumgebungen ausgelegte Scripting-Unterstützung, oder die Interaktion und Kommunikation mit externen Prozessen, umsetzbar sind. Diese Punkte sind bei der Erstellung der Anforderungen entsprechend zu beachten. Hier muss abgewogen werden, inwiefern die Einschränkungen den Nutzer limitieren könnten und die Attraktivität der neuen Oberfläche verringern.
