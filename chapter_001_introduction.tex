\chapter{Einleitung}
\label{chap:introduction}

\section{Motivation}
Ein Großteil der heute genutzten Software kann über eine Weboberfläche angesprochen werden. Dies hat den Vorteil, dass keine langwierigen und komplizierten Installationen vorgenommen werden müssen und, eine bestehende Internetverbindung vorausgesetzt, die Applikation von überall erreichbar ist. Die Firma combit GmbH hat diese Entwicklung bereits früh erkannt und bietet für den combit Relationship Manager eine entsprechende Weboberfläche an. Diese existiert bereits seit etlichen Jahren und basiert, bedingt durch den damaligen Stand der Technik auf nicht mehr zeitgemäßen Technologien. Da der Fokus der Entwicklung -- bestimmt durch die Nutzung der Kunden -- auf der Desktopanwendung liegt waren auch nicht genug Ressourcen vorhanden um die Technologien nach und nach durch modernere Alternativen zu ersetzen.
Webtechnologien haben außerdem nicht nur Vorteile für ihre Nutzer, sie bringen auch einige interessante Aspekte für das Entwicklerteam mit sich. Sehr viele technische Fortschritte fanden in den letzten Jahren in diesem Bereich statt, was einem nicht nur ermöglicht moderne Projekte und Konzepte kennen zu lernen, sondern auch, dass man eine breite Auswahl für eine solide Basis des eigenen Produkts hat. Ein spezielles Beispiel für einen solchen Fortschritt sind Programme, die es einem erlauben Webseiten wie normale Programme nativ auf einem Desktopcomputer darstellen zu können. Das Ziel hierbei ist, langfristig nur noch eine einzige Version für alle Umgebungen entwickeln zu müssen, wodurch Ressourcen gespart werden können und durch den einheitlichen Fokus, so die Erwartung, ein besseres Produkt entsteht. Dieses Potential der Webtechnologien wird dadurch noch weiter gesteigert, dass durch einfache Sprachen wie HTML, CSS und JavaScript, welches nicht kompiliert werden muss und auf so gut wie jedem modernen Gerät lauffähig ist, schnelle Iterationsschritte von Software möglich sind.
Wegen der genannten Gründe wurde entschieden, das Thema Web mit einer frischen Herangehensweise, einer neuen Perspektive und modernen Technologien neu anzugehen um den combit Relationship Manager sowohl extern als auch intern noch attraktiver zu machen.

\section{Erwartete Schwierigkeiten}
Bereits im Vorfeld soll ein kleiner Überblick über die Bereiche dieser Arbeit gegeben werden, welche im Laufe der Umsetzung zu Schwierigkeiten führen könnten. Dies dient sowohl der Kontrolle der eigenen Fähigkeit Probleme bereits im Voraus korrekt einschätzen zu können, als auch dazu Einschätzungen von Problemstellen zukünftiger Projekte zu verbessern indem am Ende der Arbeit mit den tatsächlichen problematischen Stellen verglichen wird. Weitere Informationen zu den hier aufgelisteten Punkten werden in Kapitel~\ref{chap:requirements} aufgeführt.

Eine der größten Herausforderungen wird sein, eine Umsetzung des User Interface zu finden, welche auch weiterhin genauso oder zumindest annähernd anpassbar ist wie dies momentan der Fall ist. Da das Layout dateibasiert gespeichert wird und im Browser nicht auf diese Dateien zugegriffen werden kann muss ein neues Konzept für die Persistierung erarbeitet werden, etwa die Speicherung in einer Datenbank, mithilfe dessen die Verteilung über das Netzwerk einfacher möglich ist. 
Neben dem angesprochenen Layout gibt es auch noch weitere Informationen die momentan nur dateibasiert auf dem jeweiligen Clientrechner gespeichert sind, diese müssen ebenfalls alle in ein neues Konzept überführt werden.

Um auf alle diese Informationen und die eigentlichen Daten, die dem Nutzer angezeigt werden sollen, zugreifen zu können wird eine entsprechende Schnittstelle (API) benötigt. Das Design dieser API ist kritisch, da nachfolgend nötige Änderungen sich auf jede Komponente auswirkt und ebenso Änderungen an diesen erfordert. 

Ebenfalls gilt es zu entscheiden, welche Features im Web überhaupt nicht oder nur mit zu vielen Schwierigkeiten, etwa Scripting-Unterstützung das auf eine Desktopumgebung ausgelegt ist oder die Interaktion mit anderen auf dem System laufenden Prozessen, umsetzbar ist und diese bei der Erstellung der Anforderungen entsprechend zu beachten. Hier muss abgewogen werden, wie stark der Nutzer im Vergleich zu bisher eingeschränkt wird und ob die Nutzung der neuen Oberfläche für ihn damit überhaupt attraktiv genug erscheint.

\section{Kapitelübersicht}
In Kapitel~\ref{chap:requirements} werden Anforderungen (auch anhand bisheriger Features) ermittelt.
In Kapitel~\ref{chap:technologies} werden die in Betracht gezogenen Technologien und deren Zusammenspiel untereinander vorgestellt. Dazu gehören die zu nutzende Programmierumgebung (Sprache, Testframework, Continuous Integration), der Vergleich verschiedener Frontend-Frameworks die anhand der in Kapitel~\ref{chap:requirements} formulierten Anforderungen bewertet werden und die Protokolle zur Kommunikation mit dem Backend.
In Kapitel~\ref{chap:concept} wird die Ausarbeitung des Konzepts mit den in Kapitel~\ref{chap:technologies} ausgesuchten Technologien und der Aufbau der API zur Kommunikation mit dem Backend beschrieben.
In Kapitel~\ref{chap:implementation} wird der Entwicklungsprozess der beispielhaften Umsetzung des erstellten Konzepts dargestellt.
In Kapitel~\ref{chap:conclusion} wird das Ergebnis kritisch hinterfragt, die tatsächlich aufgetretenen Schwierigkeiten mit den zuvor Erwarteten verglichen und ein Fazit gezogen.
