\chapter{Einleitung}

% was (+ wieso web ansicht -> vorteile im vergleich zu bisherigem ansatz und inwiefern diese den aufwand rechtfertigen)

\section{Motivation}

\section{Voraussichtliche Schwierigkeiten}
Automatische Umsetzung, Scripting, ...
- framework mit dem automatische umsetzung von UI (+ von kunden weiterhin möglichst einfach individualisierbar), entwicklung / wartung, testbarkeit möglichst einfach ist
- kommunikation zwischen UI und bisherigem Programm
- anforderungen (scripting, filter, ...)

\section{Kapitelübersicht}
In Kapitel 2 \fixme{link} werden Anforderungen (auch anhand bisheriger Features) ermittelt.
In Kapitel 3 \fixme{link} werden verschiedene Frontend-Frameworks verglichen und anhand der Anforderungen das geeignetste rausgesucht.
In Kapitel \fixme{neues Kapitel oder in 3 aufnehmen und umbenennen} wird diskutiert welches Protokoll zum Austausch der Daten genutzt wird (REST / GraphQL), wie die API aussehen soll (Endpoints + Aufbau der Nachrichten).
In Kapitel \fixme{neues Kapitel} wird die Ausarbeitung des Konzepts beschrieben.
In Kapitel 4 \fixme{link} wird der Entwicklungsprozess der beispielhaften Umsetzung des erstellten Konzepts dargestellt.
In Kapitel 5 \fixme{link} wird das Ergebnis kritisch hinterfragt, die tatsächlich aufgetretenen Schwierigkeiten mit den vorausgesagten verglichen und ein Fazit gezogen.
