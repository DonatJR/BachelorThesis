\chapter{Zusammenfassung (Abstract)}

Webapplikationen in Form von mobilen Apps, Single-Page-Applikationen, \ldots \fixme{beispiel} machen heute bereits einen sehr großen Teil häufig genutzter Software aus.
In Zukunft wird der Anteil von webbasierter Software im Vergleich zu traditioneller Software noch weiter ansteigen (\fixme{quelle?}).
Auch für Unternehmen ist es langfristig wichtig diesem Trend zu folgen um Kunden flexiblere Softwarelösungen anbieten zu können und nicht den Eindruck zu erwecken man könne nicht mit modernen Entwicklungen mithalten (\fixme{formulierung}).
Die Firma combit GmbH entwickelt mit dem combit Relationship Manager eine Software für das Management von Kundenbeziehungen (Customer Relationship Management), die zum momentanen Zeitpunkt aus einer Desktopapplikation, einer Webapplikation und einer mobile Webapplikation besteht, welche unabhängig voneinander entwickelt und benutzt werden.
Im Rahmen dieser Bachelorthesis soll ein Architekturkonzept für eine einheitliche Oberfläche (ThinClient\fixme{footnote?}) basierend auf moderner Webtechnologie erarbeitet werden, mithilfe derer die drei momentanen Ausführungen des combit Relationship Managers abgelöst werden können.

Zur Erstellung dieses Konzepts beschäftigt sich diese Arbeit unter anderem mit der Evaluierung eines geeigneten Frontend-Frameworks bzw.\ einer Frontend-Bibliothek, der Kommunikation zwischen neuem UI-Layer und Backend und der Unterstützung aller bisherigen Features der Desktopapplikation.
Zwei besonders spannende Probleme existieren hier zum einen in der Flexibilität der momentanen Oberfläche, welche von jedem Nutzer individuell anpassbar ist. Diese Flexibilität soll durch eine automatische Konvertierung zu einem von der neuen UI darstellbarem Format und weiterhin anpassbarem Format erhalten bleiben. 
Zum anderen bietet die Desktopapplikation die Möglichkeit Skripte und automatisierte Aufgaben zur Anpassung der zugrunde liegenden Daten, für Darstellungsbedingungen und weitere Zwecke direkt auf dem Client auszuführen. Das Ausführen auf dem Client ist bei Webapplikationen nicht ohne Weiteres möglich, weshalb eine andere Lösung für diese Funktionalität gefunden werden muss.
