\chapter{Zusammenfassung (Abstract)}

Webapplikationen in Form von mobilen Apps, Single-Page-Applikationen, \ldots \fixme{beispiel} machen heute bereits einen sehr großen Teil häufig genutzter Software aus\fixme{quelle}.
In Zukunft wird der Anteil webbasierter Software im Vergleich zu traditioneller Software noch weiter ansteigen (\fixme{quelle}).
Auch für Unternehmen ist es langfristig unabdingbar, diesem Trend zu folgen, um Kunden flexiblere Softwarelösungen zur Verfügung zu stellen und nicht den Eindruck zu erwecken, man könne nicht mit modernen Entwicklungen Schritt halten.
Die Firma \textit{combit GmbH} entwickelt mit dem \textit{\gls{crm}} eine Software für das Management von Kundenbeziehungen (\gls{CRM}), die zum gegenwärtigen Zeitpunkt aus einer Desktopapplikation, einer Webapplikation und einer mobilen Webapplikation besteht, welche unabhängig voneinander entwickelt sowie benutzt werden.
Im Rahmen dieser Bachelorthesis soll ein Architekturkonzept für eine einheitliche Oberfläche (ThinClient\fixme{footnote?}) basierend auf moderner Webtechnologie erarbeitet werden, mithilfe derer die drei aktuellen Ausführungen des \textit{\gls{crm}} langfristig abgelöst werden können.

Zur Erstellung dieses Konzepts beschäftigt sich diese Arbeit unter anderem mit der Evaluierung eines geeigneten Frontend-Frameworks bzw.\ einer Frontend-Bibliothek, der Kommunikation zwischen neuem UI-Layer und Backend sowie der Unterstützung aller bisherigen Features der Desktopapplikation.
Zwei besonders spannende Herausforderungen ergeben sich aus dem Anspruch der Beibehaltung der Flexibilität der momentanen Oberfläche, welche von jedem Nutzer individuell anpassbar ist. Diese Flexibilität soll durch eine automatische Konvertierung zu einem von der neuen UI darstellbaren und weiterhin anpassbaren Format erhalten bleiben. Zum Anderen ist es aber auch wichtig webspezifische Anforderungen, vor allem ein responsives Layout und Design, mit der Flexibilität der Oberfläche zu vereinen.
